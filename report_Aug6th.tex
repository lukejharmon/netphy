\documentclass[12pt,]{article}
\usepackage{lmodern}
\usepackage{amssymb,amsmath}
\usepackage{ifxetex,ifluatex}
\usepackage{fixltx2e} % provides \textsubscript
\ifnum 0\ifxetex 1\fi\ifluatex 1\fi=0 % if pdftex
  \usepackage[T1]{fontenc}
  \usepackage[utf8]{inputenc}
\else % if luatex or xelatex
  \ifxetex
    \usepackage{mathspec}
  \else
    \usepackage{fontspec}
  \fi
  \defaultfontfeatures{Ligatures=TeX,Scale=MatchLowercase}
\fi
% use upquote if available, for straight quotes in verbatim environments
\IfFileExists{upquote.sty}{\usepackage{upquote}}{}
% use microtype if available
\IfFileExists{microtype.sty}{%
\usepackage{microtype}
\UseMicrotypeSet[protrusion]{basicmath} % disable protrusion for tt fonts
}{}
\usepackage[margin=1in]{geometry}
\usepackage{hyperref}
\hypersetup{unicode=true,
            pdftitle={Simulations for Harmon \& Nuismer},
            pdfborder={0 0 0},
            breaklinks=true}
\urlstyle{same}  % don't use monospace font for urls
\usepackage{color}
\usepackage{fancyvrb}
\newcommand{\VerbBar}{|}
\newcommand{\VERB}{\Verb[commandchars=\\\{\}]}
\DefineVerbatimEnvironment{Highlighting}{Verbatim}{commandchars=\\\{\}}
% Add ',fontsize=\small' for more characters per line
\usepackage{framed}
\definecolor{shadecolor}{RGB}{248,248,248}
\newenvironment{Shaded}{\begin{snugshade}}{\end{snugshade}}
\newcommand{\KeywordTok}[1]{\textcolor[rgb]{0.13,0.29,0.53}{\textbf{{#1}}}}
\newcommand{\DataTypeTok}[1]{\textcolor[rgb]{0.13,0.29,0.53}{{#1}}}
\newcommand{\DecValTok}[1]{\textcolor[rgb]{0.00,0.00,0.81}{{#1}}}
\newcommand{\BaseNTok}[1]{\textcolor[rgb]{0.00,0.00,0.81}{{#1}}}
\newcommand{\FloatTok}[1]{\textcolor[rgb]{0.00,0.00,0.81}{{#1}}}
\newcommand{\ConstantTok}[1]{\textcolor[rgb]{0.00,0.00,0.00}{{#1}}}
\newcommand{\CharTok}[1]{\textcolor[rgb]{0.31,0.60,0.02}{{#1}}}
\newcommand{\SpecialCharTok}[1]{\textcolor[rgb]{0.00,0.00,0.00}{{#1}}}
\newcommand{\StringTok}[1]{\textcolor[rgb]{0.31,0.60,0.02}{{#1}}}
\newcommand{\VerbatimStringTok}[1]{\textcolor[rgb]{0.31,0.60,0.02}{{#1}}}
\newcommand{\SpecialStringTok}[1]{\textcolor[rgb]{0.31,0.60,0.02}{{#1}}}
\newcommand{\ImportTok}[1]{{#1}}
\newcommand{\CommentTok}[1]{\textcolor[rgb]{0.56,0.35,0.01}{\textit{{#1}}}}
\newcommand{\DocumentationTok}[1]{\textcolor[rgb]{0.56,0.35,0.01}{\textbf{\textit{{#1}}}}}
\newcommand{\AnnotationTok}[1]{\textcolor[rgb]{0.56,0.35,0.01}{\textbf{\textit{{#1}}}}}
\newcommand{\CommentVarTok}[1]{\textcolor[rgb]{0.56,0.35,0.01}{\textbf{\textit{{#1}}}}}
\newcommand{\OtherTok}[1]{\textcolor[rgb]{0.56,0.35,0.01}{{#1}}}
\newcommand{\FunctionTok}[1]{\textcolor[rgb]{0.00,0.00,0.00}{{#1}}}
\newcommand{\VariableTok}[1]{\textcolor[rgb]{0.00,0.00,0.00}{{#1}}}
\newcommand{\ControlFlowTok}[1]{\textcolor[rgb]{0.13,0.29,0.53}{\textbf{{#1}}}}
\newcommand{\OperatorTok}[1]{\textcolor[rgb]{0.81,0.36,0.00}{\textbf{{#1}}}}
\newcommand{\BuiltInTok}[1]{{#1}}
\newcommand{\ExtensionTok}[1]{{#1}}
\newcommand{\PreprocessorTok}[1]{\textcolor[rgb]{0.56,0.35,0.01}{\textit{{#1}}}}
\newcommand{\AttributeTok}[1]{\textcolor[rgb]{0.77,0.63,0.00}{{#1}}}
\newcommand{\RegionMarkerTok}[1]{{#1}}
\newcommand{\InformationTok}[1]{\textcolor[rgb]{0.56,0.35,0.01}{\textbf{\textit{{#1}}}}}
\newcommand{\WarningTok}[1]{\textcolor[rgb]{0.56,0.35,0.01}{\textbf{\textit{{#1}}}}}
\newcommand{\AlertTok}[1]{\textcolor[rgb]{0.94,0.16,0.16}{{#1}}}
\newcommand{\ErrorTok}[1]{\textcolor[rgb]{0.64,0.00,0.00}{\textbf{{#1}}}}
\newcommand{\NormalTok}[1]{{#1}}
\usepackage{graphicx,grffile}
\makeatletter
\def\maxwidth{\ifdim\Gin@nat@width>\linewidth\linewidth\else\Gin@nat@width\fi}
\def\maxheight{\ifdim\Gin@nat@height>\textheight\textheight\else\Gin@nat@height\fi}
\makeatother
% Scale images if necessary, so that they will not overflow the page
% margins by default, and it is still possible to overwrite the defaults
% using explicit options in \includegraphics[width, height, ...]{}
\setkeys{Gin}{width=\maxwidth,height=\maxheight,keepaspectratio}
\IfFileExists{parskip.sty}{%
\usepackage{parskip}
}{% else
\setlength{\parindent}{0pt}
\setlength{\parskip}{6pt plus 2pt minus 1pt}
}
\setlength{\emergencystretch}{3em}  % prevent overfull lines
\providecommand{\tightlist}{%
  \setlength{\itemsep}{0pt}\setlength{\parskip}{0pt}}
\setcounter{secnumdepth}{0}
% Redefines (sub)paragraphs to behave more like sections
\ifx\paragraph\undefined\else
\let\oldparagraph\paragraph
\renewcommand{\paragraph}[1]{\oldparagraph{#1}\mbox{}}
\fi
\ifx\subparagraph\undefined\else
\let\oldsubparagraph\subparagraph
\renewcommand{\subparagraph}[1]{\oldsubparagraph{#1}\mbox{}}
\fi
\usepackage{setspace}
\usepackage{lineno}
\usepackage[font=scriptsize,labelfont=bf]{caption}
\usepackage{indentfirst}

%%% Use protect on footnotes to avoid problems with footnotes in titles
\let\rmarkdownfootnote\footnote%
\def\footnote{\protect\rmarkdownfootnote}

%%% Change title format to be more compact
\usepackage{titling}

% Create subtitle command for use in maketitle
\newcommand{\subtitle}[1]{
  \posttitle{
    \begin{center}\large#1\end{center}
    }
}

\setlength{\droptitle}{-2em}
  \title{Simulations for Harmon \& Nuismer}
  \pretitle{\vspace{\droptitle}\centering\huge}
  \posttitle{\par}
  \author{}
  \preauthor{}\postauthor{}
  \predate{\centering\large\emph}
  \postdate{\par}
  \date{August 6th, 2016}

\begin{document}
\maketitle

\linenumbers
\pagenumbering{gobble} \setlength{\parindent}{1.0cm} \raggedright

\section{Symmetrical transition
rates}\label{symmetrical-transition-rates}

Simulation parameters:

\begin{Shaded}
\begin{Highlighting}[]
\NormalTok{pSpec <-}\StringTok{ }\KeywordTok{runif}\NormalTok{(}\DataTypeTok{n =} \DecValTok{1000}\NormalTok{, }\DataTypeTok{min =} \DecValTok{0}\NormalTok{, }\DataTypeTok{max =} \DecValTok{2}\NormalTok{)}
\NormalTok{q01 <-}\StringTok{ }\KeywordTok{runif}\NormalTok{(}\DataTypeTok{n =} \DecValTok{1000}\NormalTok{, }\DataTypeTok{min =} \DecValTok{0}\NormalTok{, }\DataTypeTok{max =} \DecValTok{1}\NormalTok{)}
\NormalTok{q10 <-}\StringTok{ }\NormalTok{q01}
\NormalTok{ntaxa <-}\StringTok{ }\KeywordTok{runif}\NormalTok{(}\DataTypeTok{n =} \DecValTok{1000}\NormalTok{, }\DataTypeTok{min =} \DecValTok{10}\NormalTok{, }\DataTypeTok{max =} \DecValTok{100}\NormalTok{)}
\NormalTok{lambda <-}\StringTok{ }\KeywordTok{runif}\NormalTok{(}\DataTypeTok{n =} \DecValTok{1000}\NormalTok{, }\DataTypeTok{min =} \DecValTok{0}\NormalTok{, }\DataTypeTok{max =} \DecValTok{2}\NormalTok{)}
\end{Highlighting}
\end{Shaded}

\begin{verbatim}
## Warning: Removed 112 rows containing missing values (geom_point).
\end{verbatim}

\begin{verbatim}
## Warning: Removed 108 rows containing missing values (geom_point).
\end{verbatim}

\begin{verbatim}
## Warning: Removed 44 rows containing missing values (geom_point).
\end{verbatim}

\begin{verbatim}
## Warning: Removed 33 rows containing missing values (geom_point).
\end{verbatim}

\begin{verbatim}
## Warning: Removed 204 rows containing missing values (geom_point).
\end{verbatim}

\begin{verbatim}
## Warning: Removed 137 rows containing missing values (geom_point).
\end{verbatim}

\begin{figure}[htbp]
\centering
\includegraphics{report_Aug6th_files/figure-latex/unnamed-chunk-3-1.pdf}
\caption{A - Fitted q01 by Simulated q01; B - Standardized (Fitted q01 -
Simulated q01) by Number of Taxa (values above 0 indicate
overestimation, and below 0 indicate underestimation); C - Fitted q10 by
Simulated q10; D - Standardized (Fitted q01 - Simulated q01) by Number
of Taxa (values above 0 indicate overestimation, and below 0 indicate
underestimation); E - Fitted pSpec by Simulated pSpec; F - Standardized
(Fitted pSpec - Simulated pSpec) by Number of Taxa (values above 0
indicate overestimation, and below 0 indicate underestimation)}
\end{figure}

\newpage

\section{Asymmetrical transition
rates}\label{asymmetrical-transition-rates}

Simulation parameters:

\begin{Shaded}
\begin{Highlighting}[]
\NormalTok{pSpec <-}\StringTok{ }\KeywordTok{runif}\NormalTok{(}\DataTypeTok{n =} \DecValTok{1000}\NormalTok{, }\DataTypeTok{min =} \DecValTok{0}\NormalTok{, }\DataTypeTok{max =} \DecValTok{2}\NormalTok{)}
\NormalTok{q01 <-}\StringTok{ }\KeywordTok{runif}\NormalTok{(}\DataTypeTok{n =} \DecValTok{1000}\NormalTok{, }\DataTypeTok{min =} \DecValTok{0}\NormalTok{, }\DataTypeTok{max =} \DecValTok{1}\NormalTok{)}
\NormalTok{q10 <-}\StringTok{ }\NormalTok{q01}
\NormalTok{ntaxa <-}\StringTok{ }\KeywordTok{runif}\NormalTok{(}\DataTypeTok{n =} \DecValTok{1000}\NormalTok{, }\DataTypeTok{min =} \DecValTok{10}\NormalTok{, }\DataTypeTok{max =} \DecValTok{100}\NormalTok{)}
\NormalTok{lambda <-}\StringTok{ }\KeywordTok{runif}\NormalTok{(}\DataTypeTok{n =} \DecValTok{1000}\NormalTok{, }\DataTypeTok{min =} \DecValTok{0}\NormalTok{, }\DataTypeTok{max =} \DecValTok{2}\NormalTok{)}
\end{Highlighting}
\end{Shaded}

\begin{verbatim}
## Warning: Removed 130 rows containing missing values (geom_point).
\end{verbatim}

\begin{verbatim}
## Warning: Removed 126 rows containing missing values (geom_point).
\end{verbatim}

\begin{verbatim}
## Warning: Removed 45 rows containing missing values (geom_point).
\end{verbatim}

\begin{verbatim}
## Warning: Removed 62 rows containing missing values (geom_point).
\end{verbatim}

\begin{verbatim}
## Warning: Removed 233 rows containing missing values (geom_point).
\end{verbatim}

\begin{verbatim}
## Warning: Removed 145 rows containing missing values (geom_point).
\end{verbatim}

\begin{figure}[htbp]
\centering
\includegraphics{report_Aug6th_files/figure-latex/unnamed-chunk-5-1.pdf}
\caption{A - Fitted q01 by Simulated q01; B - Standardized (Fitted q01 -
Simulated q01) by Number of Taxa (values above 0 indicate
overestimation, and below 0 indicate underestimation); C - Fitted q10 by
Simulated q10; D - Standardized (Fitted q01 - Simulated q01) by Number
of Taxa (values above 0 indicate overestimation, and below 0 indicate
underestimation); E - Fitted pSpec by Simulated pSpec; F - Standardized
(Fitted pSpec - Simulated pSpec) by Number of Taxa (values above 0
indicate overestimation, and below 0 indicate underestimation)}
\end{figure}


\end{document}
